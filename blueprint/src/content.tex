\chapter{Current Formalization (Completed)}

\section{Parameterization and Ambient Field}

\begin{definition}[Quadratic parameter]\label{def:isQuadraticParam}
A parameter $d \in \Z$ is admissible when $d \neq 0$, $d \neq 1$, and $d$ is squarefree.
\lean{../ClassificationOfIntegersOfQuadraticNumberFields/Base.lean}
\leanok
\end{definition}

\begin{definition}[Ambient quadratic algebra]\label{def:Qsqrtd}
For admissible $d$, we work in $\Q(\sqrt{d})$ modeled by `QuadraticAlgebra`.
\lean{../ClassificationOfIntegersOfQuadraticNumberFields/Base.lean}
\leanok
\uses{def:isQuadraticParam}
\end{definition}

\begin{lemma}[Nonsquare transfer]\label{lem:not_isSquare_int}
Squarefree admissible parameters are not integer squares.
\lean{../ClassificationOfIntegersOfQuadraticNumberFields/Base.lean}
\leanok
\uses{def:isQuadraticParam}
\end{lemma}

\section{Trace, Norm, and Half-Integral Coordinates}

\begin{theorem}[Trace formula]\label{thm:trace_eq_two_re}
For $x \in \Q(\sqrt{d})$, $\Tr(x)=2\,\mathrm{re}(x)$.
\lean{../ClassificationOfIntegersOfQuadraticNumberFields/MinimalPolynomial.lean}
\leanok
\end{theorem}

\begin{theorem}[Norm formula]\label{thm:norm_eq_sqr_minus_d_sqr}
For $x \in \Q(\sqrt{d})$, $\Nm(x)=\mathrm{re}(x)^2-d\,\mathrm{im}(x)^2$.
\lean{../ClassificationOfIntegersOfQuadraticNumberFields/MinimalPolynomial.lean}
\leanok
\end{theorem}

\begin{definition}[Half-integral point]\label{def:halfInt}
Define $\frac{a' + b'\sqrt d}{2}$ as an element of $\Q(\sqrt d)$.
\lean{../ClassificationOfIntegersOfQuadraticNumberFields/HalfInt.lean}
\leanok
\end{definition}

\begin{theorem}[Trace and norm of half-integral elements]\label{thm:trace_norm_halfInt}
The project proves explicit trace and norm formulas for `halfInt`.
\lean{../ClassificationOfIntegersOfQuadraticNumberFields/HalfInt.lean}
\leanok
\uses{def:halfInt, thm:trace_eq_two_re, thm:norm_eq_sqr_minus_d_sqr}
\end{theorem}

\section{Parity and Mod-$4$ Criteria}

\begin{theorem}[Main mod-$4$ dichotomy]\label{thm:mod4_dichotomy}
$4 \mid a'^2 - d b'^2$ iff either both numerators are even, or both odd with $d \equiv 1 \pmod 4$.
\lean{../ClassificationOfIntegersOfQuadraticNumberFields/ModFourCriteria.lean}
\leanok
\end{theorem}

\begin{corollary}[Non-$1$ mod $4$ branch]\label{cor:mod4_ne_one}
If $d \not\equiv 1 \pmod 4$, then $4 \mid a'^2 - d b'^2$ iff $a',b'$ are both even.
\lean{../ClassificationOfIntegersOfQuadraticNumberFields/ModFourCriteria.lean}
\leanok
\uses{thm:mod4_dichotomy}
\end{corollary}

\section{Embedding from $\Z[\sqrt d]$ and Completed Classification Branch}

\begin{definition}[Canonical embedding]\label{def:zsqrtdToQsqrtd}
The ring map $\Z[\sqrt d] \to \Q(\sqrt d)$ is defined and shown injective.
\lean{../ClassificationOfIntegersOfQuadraticNumberFields/ClassificationToZsqrtd.lean}
\leanok
\end{definition}

\begin{theorem}[Image criterion for `halfInt`]\label{thm:halfInt_range_even_even}
A half-integral element lies in the image of $\Z[\sqrt d]$ iff both numerators are even.
\lean{../ClassificationOfIntegersOfQuadraticNumberFields/ClassificationToZsqrtd.lean}
\leanok
\uses{def:halfInt, def:zsqrtdToQsqrtd}
\end{theorem}

\begin{theorem}[Completed branch of integrality classification]\label{thm:classification_ne_one_mod_four}
For squarefree $d \not\equiv 1 \pmod 4$, the divisibility condition is equivalent to representability from $\Z[\sqrt d]$.
\lean{../ClassificationOfIntegersOfQuadraticNumberFields/ClassificationToZsqrtd.lean}
\leanok
\uses{cor:mod4_ne_one, thm:halfInt_range_even_even}
\end{theorem}

\section{Field Isomorphism Rigidity}

\begin{theorem}[Distinct parameters give non-isomorphic fields]\label{thm:quadratic_fields_not_iso}
For distinct admissible squarefree parameters $d_1 \neq d_2$, one has
$\Q(\sqrt{d_1}) \not\simeq_{\Q} \Q(\sqrt{d_2})$.
\lean{../ClassificationOfIntegersOfQuadraticNumberFields/NonIso.lean}
\leanok
\uses{def:isQuadraticParam}
\end{theorem}
\chapter{Roadmap I: `ClassificationToZsqrtd.lean` TODO List}

\begin{theorem}[Future theorem `halfInt_mem_range_halfOrder_iff_same_parity`]\label{thm:future_halfInt_mem_range_halfOrder_iff_same_parity}
For $d \equiv 1 \pmod 4$, characterize when `halfInt d a' b'` lies in the image of
$\Z\!\left[\frac{1+\sqrt d}{2}\right]$ by the parity condition $a' \equiv b' \pmod 2$.
\notready
\uses{thm:mod4_dichotomy, def:halfInt}
\discussion{Planned target in `ClassificationToZsqrtd.lean` for the missing $d \equiv 1 \pmod 4$ branch.}
\end{theorem}

\begin{theorem}[Future theorem `dvd_four_sub_sq_iff_exists_halfOrder_image_of_one_mod_four`]\label{thm:future_dvd_four_sub_sq_iff_exists_halfOrder_image_of_one_mod_four}
Complete the one-mod-four branch by expressing
$4 \mid (a'^2 - d b'^2)$ iff representability via $\Z\!\left[\frac{1+\sqrt d}{2}\right]$.
\notready
\uses{thm:future_halfInt_mem_range_halfOrder_iff_same_parity}
\end{theorem}

\begin{theorem}[Future theorem `dvd_four_sub_sq_iff_exists_integral_model_image`]\label{thm:future_dvd_four_sub_sq_iff_exists_integral_model_image}
Unify both branches into one image-theoretic criterion matching the full ring-of-integers classification model.
\notready
\uses{thm:classification_ne_one_mod_four, thm:future_dvd_four_sub_sq_iff_exists_halfOrder_image_of_one_mod_four}
\end{theorem}

\chapter{Roadmap II: `ModFourCriteria.lean` Refinements}

\begin{theorem}[Future theorem `same_parity_iff_exists_half_order_coordinates`]\label{thm:future_same_parity_iff_exists_half_order_coordinates}
Extract a standalone parity-to-coordinates equivalence specialized to $d \equiv 1 \pmod 4$ for reuse in classification proofs.
\notready
\uses{thm:mod4_dichotomy, thm:future_halfInt_mem_range_halfOrder_iff_same_parity}
\end{theorem}

\begin{theorem}[Future theorem `dvd_four_sub_sq_iff_half_order_coordinate_condition`]\label{thm:future_dvd_four_sub_sq_iff_half_order_coordinate_condition}
Package the mod-$4$ criterion in a form directly consumable by `ClassificationToZsqrtd.lean` without case-by-case rewrites.
\notready
\uses{thm:future_same_parity_iff_exists_half_order_coordinates}
\end{theorem}

\chapter{Roadmap III: `Base.lean` Transport and Normalization}

\begin{theorem}[Future theorem `integrality_transport_via_rescale`]\label{thm:future_integrality_transport_via_rescale}
Transport integrality predicates along `Qsqrtd.rescale` to compare equivalent square-class parameters.
\notready
\uses{thm:future_dvd_four_sub_sq_iff_exists_integral_model_image}
\end{theorem}

\begin{theorem}[Future theorem `classification_invariant_under_square_scaling`]\label{thm:future_classification_invariant_under_square_scaling}
Show the classification theorem is invariant under replacing $d$ by $a^2 d$ (with $a \in \Q^\times$).
\notready
\uses{thm:future_integrality_transport_via_rescale}
\end{theorem}

\chapter{Roadmap IV: `NonIso.lean` and Canonical Parameter API}

\begin{theorem}[Future theorem `quadratic_param_canonical_of_field_iso`]\label{thm:future_quadratic_param_canonical_of_field_iso}
Combine rigidity with completed classification to expose a canonical-parameter interface for quadratic fields.
\notready
\uses{thm:quadratic_fields_not_iso, thm:future_dvd_four_sub_sq_iff_exists_integral_model_image}
\end{theorem}

\begin{theorem}[Future theorem `ring_of_integers_classification_isomorphism_invariant`]\label{thm:future_ring_of_integers_classification_isomorphism_invariant}
State explicitly that the ring-of-integers classification API is preserved under field isomorphisms of quadratic fields.
\notready
\uses{thm:future_quadratic_param_canonical_of_field_iso, thm:future_classification_invariant_under_square_scaling}
\end{theorem}
