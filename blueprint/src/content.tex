\chapter{Current Formalization (Completed)}

\section{Parameterization and Ambient Field}

\begin{definition}[Quadratic parameter]\label{def:isQuadraticParam}
A parameter $d \in \Z$ is admissible when $d \neq 0$, $d \neq 1$, and $d$ is squarefree.
\lean{../ClassificationOfIntegersOfQuadraticNumberFields/Base.lean}
\leanok
\end{definition}

\begin{definition}[Ambient quadratic algebra]\label{def:Qsqrtd}
For admissible $d$, we work in $\Q(\sqrt{d})$ modeled by `QuadraticAlgebra`.
\lean{../ClassificationOfIntegersOfQuadraticNumberFields/Base.lean}
\leanok
\uses{def:isQuadraticParam}
\end{definition}

\begin{lemma}[Nonsquare transfer]\label{lem:not_isSquare_int}
Squarefree admissible parameters are not integer squares.
\lean{../ClassificationOfIntegersOfQuadraticNumberFields/Base.lean}
\leanok
\uses{def:isQuadraticParam}
\end{lemma}

\section{Trace, Norm, and Half-Integral Coordinates}

\begin{theorem}[Trace formula]\label{thm:trace_eq_two_re}
For $x \in \Q(\sqrt{d})$, $\Tr(x)=2\,\mathrm{re}(x)$.
\lean{../ClassificationOfIntegersOfQuadraticNumberFields/MinimalPolynomial.lean}
\leanok
\end{theorem}

\begin{theorem}[Norm formula]\label{thm:norm_eq_sqr_minus_d_sqr}
For $x \in \Q(\sqrt{d})$, $\Nm(x)=\mathrm{re}(x)^2-d\,\mathrm{im}(x)^2$.
\lean{../ClassificationOfIntegersOfQuadraticNumberFields/MinimalPolynomial.lean}
\leanok
\end{theorem}

\begin{definition}[Half-integral point]\label{def:halfInt}
Define $\frac{a' + b'\sqrt d}{2}$ as an element of $\Q(\sqrt d)$.
\lean{../ClassificationOfIntegersOfQuadraticNumberFields/HalfInt.lean}
\leanok
\end{definition}

\begin{theorem}[Trace and norm of half-integral elements]\label{thm:trace_norm_halfInt}
The project proves explicit trace and norm formulas for `halfInt`.
\lean{../ClassificationOfIntegersOfQuadraticNumberFields/HalfInt.lean}
\leanok
\uses{def:halfInt, thm:trace_eq_two_re, thm:norm_eq_sqr_minus_d_sqr}
\end{theorem}

\section{Parity and Mod-$4$ Criteria}

\begin{theorem}[Main mod-$4$ dichotomy]\label{thm:mod4_dichotomy}
$4 \mid a'^2 - d b'^2$ iff either both numerators are even, or both odd with $d \equiv 1 \pmod 4$.
\lean{../ClassificationOfIntegersOfQuadraticNumberFields/ModFourCriteria.lean}
\leanok
\end{theorem}

\begin{corollary}[Non-$1$ mod $4$ branch]\label{cor:mod4_ne_one}
If $d \not\equiv 1 \pmod 4$, then $4 \mid a'^2 - d b'^2$ iff $a',b'$ are both even.
\lean{../ClassificationOfIntegersOfQuadraticNumberFields/ModFourCriteria.lean}
\leanok
\uses{thm:mod4_dichotomy}
\end{corollary}

\section{Embedding from $\Z[\sqrt d]$ and Completed Classification Branch}

\begin{definition}[Canonical embedding]\label{def:zsqrtdToQsqrtd}
The ring map $\Z[\sqrt d] \to \Q(\sqrt d)$ is defined and shown injective.
\lean{../ClassificationOfIntegersOfQuadraticNumberFields/ClassificationToZsqrtd.lean}
\leanok
\end{definition}

\begin{theorem}[Image criterion for `halfInt`]\label{thm:halfInt_range_even_even}
A half-integral element lies in the image of $\Z[\sqrt d]$ iff both numerators are even.
\lean{../ClassificationOfIntegersOfQuadraticNumberFields/ClassificationToZsqrtd.lean}
\leanok
\uses{def:halfInt, def:zsqrtdToQsqrtd}
\end{theorem}

\begin{theorem}[Completed branch of integrality classification]\label{thm:classification_ne_one_mod_four}
For squarefree $d \not\equiv 1 \pmod 4$, the divisibility condition is equivalent to representability from $\Z[\sqrt d]$.
\lean{../ClassificationOfIntegersOfQuadraticNumberFields/ClassificationToZsqrtd.lean}
\leanok
\uses{cor:mod4_ne_one, thm:halfInt_range_even_even}
\end{theorem}

\section{Field Isomorphism Rigidity}

\begin{theorem}[Distinct parameters give non-isomorphic fields]\label{thm:quadratic_fields_not_iso}
For distinct admissible squarefree parameters $d_1 \neq d_2$, one has
$\Q(\sqrt{d_1}) \not\simeq_{\Q} \Q(\sqrt{d_2})$.
\lean{../ClassificationOfIntegersOfQuadraticNumberFields/NonIso.lean}
\leanok
\uses{def:isQuadraticParam}
\end{theorem}
\chapter{Roadmap I: The $d \equiv 1 \pmod 4$ Branch}

\begin{definition}[Planned: half-integer order generator]\label{def:plan_half_order}
Introduce the order generator $\omega_d := \frac{1+\sqrt d}{2}$ in $\Q(\sqrt d)$ and formalize
its basic arithmetic interface.
\notready
\uses{def:Qsqrtd, def:halfInt}
\end{definition}

\begin{theorem}[Planned: same-parity membership criterion]\label{thm:plan_same_parity_membership}
For $d \equiv 1 \pmod 4$, characterize representability in
$\Z[\omega_d]$ by the same-parity condition on numerators.
\notready
\uses{thm:mod4_dichotomy, def:plan_half_order}
\end{theorem}

\begin{theorem}[Planned: completed $1$-mod-$4$ classification branch]\label{thm:plan_one_mod_four_branch}
Complete the branch currently left as TODO: express integrality via
$\Z\!\left[\frac{1+\sqrt d}{2}\right]$ when $d \equiv 1 \pmod 4$.
\notready
\uses{thm:plan_same_parity_membership}
\discussion{This is the missing branch mentioned in `ClassificationToZsqrtd.lean`.}
\end{theorem}
\chapter{Roadmap II: Global Ring-of-Integers Classification}

\begin{theorem}[Planned: unified statement of $\mathcal O_{\Q(\sqrt d)}$]\label{thm:plan_global_classification}
Combine the $d \not\equiv 1 \pmod 4$ and $d \equiv 1 \pmod 4$ branches into a single theorem
classifying integral elements of quadratic fields for squarefree admissible parameters.
\notready
\uses{thm:classification_ne_one_mod_four, thm:plan_one_mod_four_branch}
\end{theorem}

\begin{corollary}[Planned: explicit basis and membership test]\label{cor:plan_explicit_basis}
Extract a practical basis-and-membership API for $\mathcal O_{\Q(\sqrt d)}$ from the unified theorem.
\notready
\uses{thm:plan_global_classification}
\end{corollary}
\chapter{Roadmap III: Transport, Normalization, and Isomorphism Interfaces}

\begin{proposition}[Planned: transport of integrality along rescaling]\label{prop:plan_rescale_transport}
Use the existing rescaling equivalence $\Q(\sqrt d) \simeq \Q(\sqrt{a^2 d})$ to transport integrality
and classification results to normalized representatives.
\notready
\uses{thm:plan_global_classification}
\end{proposition}

\begin{proposition}[Planned: bridge between classification and non-isomorphism]\label{prop:plan_classification_noniso_bridge}
Connect the completed classification theorem with field-rigidity lemmas to provide a coherent
parameter-canonicalization story in one API layer.
\notready
\uses{thm:quadratic_fields_not_iso, thm:plan_global_classification, prop:plan_rescale_transport}
\end{proposition}
\chapter{Roadmap IV: API Strengthening and Project Polish}

\begin{proposition}[Planned: stronger norm/trace/minpoly API]\label{prop:plan_norm_trace_api}
Add reusable lemmas around trace, norm, minimal-polynomial identities, and coercions so downstream
proofs avoid repeated low-level rewrites.
\notready
\uses{thm:trace_eq_two_re, thm:norm_eq_sqr_minus_d_sqr, thm:plan_global_classification}
\end{proposition}

\begin{proposition}[Planned: module split and theorem indexing]\label{prop:plan_module_split}
Refine file layout and theorem naming/indexing to make the library easier to extend and to align
with blueprint chapter boundaries.
\notready
\uses{prop:plan_norm_trace_api, cor:plan_explicit_basis}
\end{proposition}
