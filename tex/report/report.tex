% Copyright (c) 2026 Frankie Feng-Cheng WANG. All rights reserved.
% Repository: https://github.com/FrankieeW/ClassificationOfIntegersOfQuadraticNumberFields

\documentclass[
  12pt,
]{assignment}

\setcounter{tocdepth}{2}
\setcounter{secnumdepth}{2}

\usepackage{amsthm}
\usepackage{fontspec}
\usepackage{float}
\setmonofont{FreeMono}
\usepackage{hyperref}
\usepackage{minted}
\newmintinline[lean]{lean4}{bgcolor=white}
\newminted[leancode]{lean4}{fontsize=\footnotesize,breaklines,breakanywhere,tabsize=4,showspaces=false}
\renewcommand{\theFancyVerbLine}{\arabic{FancyVerbLine}}
\renewcommand{\FancyVerbFormatLine}[1]{#1}
\newcommand{\leancodefile}[2][]{%
  \inputminted[fontsize=\footnotesize,breaklines,breakanywhere,tabsize=4,showspaces=false,linenos=true,#1]{lean4}{#2}%
}
\usemintedstyle{tango}

\usepackage{mathtools}
\usepackage{amssymb}
\usepackage{xcolor}
\newcommand\nb{\addtocounter{equation}{1}\tag{\theequation}}
\pgfplotsset{compat=1.18}

\setlength{\headheight}{29.34845pt}
\addtolength{\topmargin}{-17.34845pt}

\title{Quadratic Integer Rings in Lean 4\\\large A Chapter-2-Style Formalization Report}
\author{Frankie Feng-Cheng WANG}
\email{maths@frankie.wang}
\github{https://github.com/FrankieeW/ClassificationOfIntegersOfQuadraticNumberFields}
\date{\today}
\institute{Department of Mathematics\\Imperial College London}
\course{MATH70040-Formalising Mathematics}
\lecturer{Dr Bhavik Mehta}

\begin{document}
\setcounter{tocdepth}{2}
\maketitle
\tableofcontents

\section{Introduction}
This report is written in the style of the algebraic-integers chapter (Lecture 2) in
George Boxer’s notes, with explicit mathematical statements for every definition,
lemma, and theorem currently formalized in the Lean project.
The mathematical scope is the quadratic field
\[
\mathbb{Q}(\sqrt d)=\{a+b\sqrt d\mid a,b\in\mathbb{Q}\},
\]
and the ring-of-integers prerequisites needed for the classification split by
\(d\bmod 4\).

\noindent\textbf{Build status (2026-02-27):}
\texttt{lake build} succeeds; no \texttt{sorry} remains in
\texttt{ClassificationOfIntegersOfQuadraticNumberFields/*.lean}.

\subsection*{Development note (motivation and current strategy)}
The current formalization strategy is intentionally pragmatic.
Historically, while preparing and submitting PR work around \texttt{Zsqrtd},
I noticed that \texttt{QuadraticAlgebra} provides a workable ambient path for this
project stage.
So for the half-integral classification workflow I currently use
\texttt{QuadraticAlgebra} and define
\[
\texttt{Qsqrtd}(d):=\texttt{QuadraticAlgebra}\;\mathbb{Q}\;(d:\mathbb{Q})\;0.
\]
This is a coarse-grained but effective bridge for now.
The reason is that the integral model needed for the
\(d\equiv 1\pmod 4\) branch, especially the
\(\mathbb{Z}[\frac{1+\sqrt{1+4k}}{2}]\)-style object, is not yet packaged as a ready
drop-in component in the way this project needs; see the related discussion:
\url{https://leanprover.zulipchat.com/#narrow/channel/217875-Is-there-code-for-X.3F/topic/Z.5B.281.2Bsqrt.281.2B4k.29.29.2F2.5D/near/520523635}.

\section{Quadratic setup and basic structures (Base.lean)}

\subsection{Definition 2.1 (quadratic parameter package)}
\textbf{Lean name:} \texttt{IsQuadraticParam}.\newline
For \(d\in\mathbb{Z}\), we define a proposition requiring
\[
d\neq 0,\qquad d\neq 1,\qquad \text{$d$ squarefree}.
\]
This is the standard canonical hypothesis for representing a quadratic field by an
integer parameter.
\leancodefile[firstline=21,lastline=26,firstnumber=21]{../../ClassificationOfIntegersOfQuadraticNumberFields/Base.lean}

\subsection{Definition 2.2 (ambient type)}
\textbf{Lean name:} \texttt{Qsqrtd}.\newline
The type
\[
\texttt{Qsqrtd}(d):=\texttt{QuadraticAlgebra}\;\mathbb{Q}\;(d: \mathbb{Q})\;0
\]
serves as the formal model of \(\mathbb{Q}(\sqrt d)\).
\leancodefile[firstline=29,lastline=29,firstnumber=29]{../../ClassificationOfIntegersOfQuadraticNumberFields/Base.lean}

\subsection{Definition 2.3 (rescaling equivalence)}
\textbf{Lean name:} \texttt{rescale}.\newline
Given \(a\in\mathbb{Q}^{\times}\), there is an algebra isomorphism
\[
\mathbb{Q}(\sqrt d)\cong\mathbb{Q}(\sqrt{a^2d}).
\]
In coordinates this is
\[
(r,s)\longmapsto (r,s a^{-1}),\qquad (r,t)\longmapsto (r,ta).
\]
\leancodefile[firstline=35,lastline=50,firstnumber=35]{../../ClassificationOfIntegersOfQuadraticNumberFields/Base.lean}

\subsection{Definition 2.4 (trace and norm abbreviations)}
\textbf{Lean names:} \texttt{trace}, \texttt{norm'}.\newline
For \(x\in \texttt{Qsqrtd}(d)\), define
\[
\mathrm{tr}(x):=x+\bar x\in\mathbb{Q},\qquad N(x):=x\bar x\in\mathbb{Q}.
\]
Lean packages these as abbreviations of the real/star and quadratic-algebra norm formulas.
\leancodefile[firstline=53,lastline=56,firstnumber=53]{../../ClassificationOfIntegersOfQuadraticNumberFields/Base.lean}

\subsection{Definition 2.5 (rational embedding)}
\textbf{Lean name:} \texttt{embed}.\newline
The canonical inclusion
\[
\mathbb{Q}\hookrightarrow \mathbb{Q}(\sqrt d)
\]
is implemented by the algebra map.
\leancodefile[firstline=59,lastline=59,firstnumber=59]{../../ClassificationOfIntegersOfQuadraticNumberFields/Base.lean}

\subsection{Definition 2.6 (nonsquare rational condition)}
\textbf{Lean name:} \texttt{IsNonsquareRat}.\newline
For integer \(d\), define
\[
\forall r\in\mathbb{Q},\quad r^2\neq d.
\]
\leancodefile[firstline=62,lastline=63,firstnumber=62]{../../ClassificationOfIntegersOfQuadraticNumberFields/Base.lean}

\subsection{Proposition 2.7 (squarefree nontrivial implies nonsquare in \(\mathbb{Z}\))}
\textbf{Lean name:} \texttt{not\_isSquare\_int}.\newline
Under \texttt{IsQuadraticParam} hypotheses,
\[
\neg\,\mathrm{IsSquare}(d)\quad\text{in }\mathbb{Z}.
\]
This excludes degeneration of the quadratic extension.
\leancodefile[firstline=66,lastline=79,firstnumber=66]{../../ClassificationOfIntegersOfQuadraticNumberFields/Base.lean}

\subsection{Proposition 2.8 (parameter hypothesis gives rational nonsquare)}
\textbf{Lean instance:} \texttt{instance (d) [IsQuadraticParam d] : IsNonsquareRat d}.\newline
From the integer nonsquare result and transfer lemmas between \(\mathbb{Z}\) and
\(\mathbb{Q}\), one obtains
\[
\forall r\in\mathbb{Q},\;r^2\neq d.
\]
\leancodefile[firstline=81,lastline=86,firstnumber=81]{../../ClassificationOfIntegersOfQuadraticNumberFields/Base.lean}

\subsection{Proposition 2.9 (field structure)}
\textbf{Lean instance:} \texttt{Field (Qsqrtd d)} under \texttt{IsNonsquareRat d}.\newline
If \(d\) is rationally nonsquare, then \(\mathbb{Q}(\sqrt d)\) is a field.
\leancodefile[firstline=88,lastline=93,firstnumber=88]{../../ClassificationOfIntegersOfQuadraticNumberFields/Base.lean}

\section{Trace, norm, and quadratic identity (MinimalPolynomial.lean)}

\subsection{Theorem 3.1 (trace formula)}
\textbf{Lean name:} \texttt{trace\_eq\_two\_re}.\newline
For \(x\in\mathbb{Q}(\sqrt d)\),
\[
\mathrm{tr}(x)=2\operatorname{Re}(x).
\]
\leancodefile[firstline=8,lastline=12,firstnumber=8]{../../ClassificationOfIntegersOfQuadraticNumberFields/MinimalPolynomial.lean}

\subsection{Theorem 3.2 (norm formula)}
\textbf{Lean name:} \texttt{norm\_eq\_sqr\_minus\_d\_sqr}.\newline
Writing \(x=a+b\sqrt d\), one has
\[
N(x)=a^2-d b^2.
\]
\leancodefile[firstline=15,lastline=19,firstnumber=15]{../../ClassificationOfIntegersOfQuadraticNumberFields/MinimalPolynomial.lean}

\subsection{Theorem 3.3 (quadratic polynomial annihilation)}
\textbf{Lean name:} \texttt{aeval\_eq\_zero\_of\_quadratic}.\newline
Each \(x\in\mathbb{Q}(\sqrt d)\) satisfies
\[
x^2-\mathrm{tr}(x)\,x+N(x)=0.
\]
This is the formal identity underlying the minimal-polynomial discussion in a quadratic
extension.
\leancodefile[firstline=8,lastline=25,firstnumber=8]{../../ClassificationOfIntegersOfQuadraticNumberFields/MinimalPolynomial.lean}

\section{Half-integral normal form (HalfInt.lean)}

\subsection{Definition 4.1}
\textbf{Lean name:} \texttt{halfInt}.\newline
For integers \(a',b',d\), define
\[
\operatorname{halfInt}(d,a',b'):=\frac{a'+b'\sqrt d}{2}\in\mathbb{Q}(\sqrt d).
\]
\leancodefile[firstline=8,lastline=9,firstnumber=8]{../../ClassificationOfIntegersOfQuadraticNumberFields/HalfInt.lean}

\subsection{Theorem 4.2 (trace of half-integral element)}
\textbf{Lean name:} \texttt{trace\_halfInt}.\newline
\[
\mathrm{tr}\!\left(\frac{a'+b'\sqrt d}{2}\right)=a'.
\]
\leancodefile[firstline=12,lastline=16,firstnumber=12]{../../ClassificationOfIntegersOfQuadraticNumberFields/HalfInt.lean}

\subsection{Theorem 4.3 (norm of half-integral element)}
\textbf{Lean name:} \texttt{norm\_halfInt}.\newline
\[
N\!\left(\frac{a'+b'\sqrt d}{2}\right)=\frac{a'^2-d b'^2}{4}.
\]
\leancodefile[firstline=18,lastline=24,firstnumber=18]{../../ClassificationOfIntegersOfQuadraticNumberFields/HalfInt.lean}

\section{Mod-4 analysis and parity classification (ModFourCriteria.lean)}
This section mirrors the key exercise pattern from Chapter 2: reduce integrality conditions
to congruence constraints.

\subsection{Lemma 5.1}
\textbf{Lean name:} \texttt{squarefree\_int\_not\_dvd\_four}.\newline
If \(d\in\mathbb{Z}\) is squarefree, then
\[
4\nmid d.
\]
\leancodefile[firstline=9,lastline=15,firstnumber=9]{../../ClassificationOfIntegersOfQuadraticNumberFields/ModFourCriteria.lean}

\subsection{Lemma 5.2}
\textbf{Lean name:} \texttt{squarefree\_int\_emod\_four}.\newline
If \(d\) is squarefree, then
\[
d\bmod 4\in\{1,2,3\}.
\]
\leancodefile[firstline=18,lastline=21,firstnumber=18]{../../ClassificationOfIntegersOfQuadraticNumberFields/ModFourCriteria.lean}

\subsection{Lemma 5.3}
\textbf{Lean name:} \texttt{Int.sq\_emod\_four\_of\_even}.\newline
If \(2\mid n\), then
\[
n^2\equiv 0\pmod 4.
\]
\leancodefile[firstline=24,lastline=27,firstnumber=24]{../../ClassificationOfIntegersOfQuadraticNumberFields/ModFourCriteria.lean}

\subsection{Lemma 5.4}
\textbf{Lean name:} \texttt{Int.sq\_emod\_four\_of\_odd}.\newline
If \(2\nmid n\), then
\[
n^2\equiv 1\pmod 4.
\]
\leancodefile[firstline=30,lastline=35,firstnumber=30]{../../ClassificationOfIntegersOfQuadraticNumberFields/ModFourCriteria.lean}

\subsection{Lemma 5.5 (internal equivalence)}
\textbf{Lean name:} \texttt{div4\_iff\_mod} (private).\newline
For integers \(a',b',d\),
\[
4\mid(a'^2-d b'^2)\iff (a'^2-d b'^2)\bmod 4=0.
\]
\leancodefile[firstline=37,lastline=39,firstnumber=37]{../../ClassificationOfIntegersOfQuadraticNumberFields/ModFourCriteria.lean}

\subsection{Theorem 5.6 (main mod-4 criterion)}
\textbf{Lean name:} \texttt{dvd\_four\_sub\_sq\_iff\_even\_even\_or\_odd\_odd\_mod\_four\_one}.\newline
Assume \(d\) squarefree. Then
\[
4\mid(a'^2-d b'^2)
\iff
\bigl(2\mid a'\ \&\ 2\mid b'\bigr)
\;\vee\;
\bigl(2\nmid a'\ \&\ 2\nmid b'\ \&\ d\equiv 1\pmod 4\bigr).
\]
\leancodefile[firstline=42,lastline=101,firstnumber=42]{../../ClassificationOfIntegersOfQuadraticNumberFields/ModFourCriteria.lean}

\subsection{Theorem 5.7 (forcing even-even when \(d\not\equiv 1\mod 4\))}
\textbf{Lean name:} \texttt{even\_even\_of\_dvd\_four\_sub\_sq\_of\_ne\_one\_mod\_four}.\newline
If \(d\) is squarefree and \(d\not\equiv 1\pmod 4\), then
\[
4\mid(a'^2-d b'^2)\implies 2\mid a'\ \text{and}\ 2\mid b'.
\]
\leancodefile[firstline=104,lastline=110,firstnumber=104]{../../ClassificationOfIntegersOfQuadraticNumberFields/ModFourCriteria.lean}

\subsection{Theorem 5.8 (equivalence in non-\(1\mod 4\) branch)}
\textbf{Lean name:} \texttt{dvd\_four\_sub\_sq\_iff\_even\_even\_of\_ne\_one\_mod\_four}.\newline
If \(d\) is squarefree and \(d\not\equiv 1\pmod 4\), then
\[
4\mid(a'^2-d b'^2)\iff \bigl(2\mid a'\ \&\ 2\mid b'\bigr).
\]
\leancodefile[firstline=113,lastline=120,firstnumber=113]{../../ClassificationOfIntegersOfQuadraticNumberFields/ModFourCriteria.lean}

\subsection{Theorem 5.9 (equivalence in \(1\mod 4\) branch)}
\textbf{Lean name:} \texttt{dvd\_four\_sub\_sq\_iff\_same\_parity\_of\_one\_mod\_four}.\newline
If \(d\) is squarefree and \(d\equiv 1\pmod 4\), then
\[
4\mid(a'^2-d b'^2)\iff a'\equiv b'\pmod 2.
\]
XV|
JH|\section{Progress and remaining work}
QK|The formalization up to this point (mod-4 classification criteria) is complete.
NR|The following sections outline the short-term and long-term development goals.
PY|
VB|\subsection{Short-term goal: Embedding equivalence}
PY|The immediate next step is to complete the embedding equivalence between
ZC|$\mathbb{Z}[\sqrt{d}]$ and $\mathbb{Q}(\sqrt{d})$:
MM|\begin{enumerate}
MM|  \item \textbf{Current state:} Two theorems remain as \texttt{sorry}:
MM|    \begin{itemize}
MM|      \item \texttt{ringOfIntegers\_equiv\_zsqrtd\_of\_mod\_four\_ne\_one}
MM|      \item \texttt{mod\_four\_ne\_one\_of\_ringOfIntegers\_equiv\_zsqrtd}
MM|    \end{itemize}
MM|  \item \textbf{Goal:} Prove both directions of the equivalence theorem
MM|    establishing the classification for the $d \not\equiv 1 \pmod 4$ case.
MM|  \item \textbf{Mathematical content:} Show that when $d \not\equiv 1 \pmod 4$,
MM|    the ring of integers $O_K$ is isomorphic to $\mathbb{Z}[\sqrt{d}]$.
MM|\end{enumerate}
PY|
NM|\subsection{Long-term goal: Refactoring and $1 \pmod 4$ classification}
ZM|The ultimate target is a complete classification of rings of integers in quadratic
ZW|fields, requiring both branches:
SQ|\begin{enumerate}
NM|  \item \textbf{Refactor \texttt{Zsqrtd} construction:}
NM|    \begin{itemize}
NM|      \item Leverage the existing \texttt{QuadraticAlgebra} infrastructure
NM|      \item Create cleaner algebraic interfaces around \texttt{Zsqrtd}
NM|      \item Improve interoperability with mathlib's quadratic field libraries
NM|    \end{itemize}
NM|  \item \textbf{Formalize $\mathbb{Z}[\frac{1+\sqrt{1+4k}}{2}]$:}
NM|    \begin{itemize}
NM|      \item Construct the integral model for the $d \equiv 1 \pmod 4$ branch
NM|      \item This corresponds to $d = 1+4k$ for $k \in \mathbb{Z}$
NM|      \item Define the ring $\mathbb{Z}[\frac{1+\sqrt{d}}{2}] \subset \mathbb{Q}(\sqrt{d})$
NM|    \end{itemize}
NM|  \item \textbf{Unified classification theorem:}
NM|    \begin{itemize}
NM|      \item Prove $O_K \cong \mathbb{Z}[\sqrt{d}]$ when $d \not\equiv 1 \pmod 4$
NM|      \item Prove $O_K \cong \mathbb{Z}[\frac{1+\sqrt{d}}{2}]$ when $d \equiv 1 \pmod 4$
NM|      \item Provide a single API-level theorem summarizing the classification
NM|    \end{itemize}
SQ|\end{enumerate}
PY|
VT|\section{Remaining (not yet formalized)}
VT|The following components are planned but not yet implemented:
NM|\begin{itemize}
NM|  \item \textbf{ClassificationToZsqrtd.lean (partial):}
NM|    \begin{itemize}
NM|      \item Section 6.1-6.5: Embedding definitions and $d \not\equiv 1 \pmod 4$ branch \emph{(complete)}
NM|      \item Section 6.6: $d \equiv 1 \pmod 4$ branch \emph{(pending)}
NM|    \end{itemize}
NM|  \item \textbf{Integral model for $d \equiv 1 \pmod 4$:}
NM|    \begin{itemize}
NM|      \item Definition of $\mathbb{Z}[\frac{1+\sqrt{d}}{2}]$
NM|      \item Proof that this ring is integrally closed in $\mathbb{Q}(\sqrt{d})$
NM|      \item Trace and norm calculations in this model
NM|    \end{itemize}
NM|  \item \textbf{Final classification theorem:}
NM|    \begin{itemize}
NM|      \item Unified statement covering both cases
NM|      \item API-level theorem with clear documentation
NM|    \end{itemize}
VT|\end{itemize}
VT|
VT|\section{Reproducibility}
KK|\begin{verbatim}
PQ|lake exe cache get
RB|lake build
MP|cd tex/report
RJ|latexmk -xelatex -shell-escape -interaction=nonstopmode -halt-on-error -output-directory=out report.tex
RR|\end{verbatim}
XY|
PT|\end{document}

\section{Embedding into \texorpdfstring{$\mathbb{Q}(\sqrt d)$}{Q(sqrt d)} and image characterization (ClassificationToZsqrtd.lean)}

\subsection{Definition 6.1 (canonical embedding)}
\textbf{Lean name:} \texttt{zsqrtdToQsqrtd}.\newline
Define the ring map
\[
\iota_d:\mathbb{Z}[\sqrt d]\longrightarrow \mathbb{Q}(\sqrt d),
\qquad m+n\sqrt d\longmapsto m+n\sqrt d,
\]
with coefficients interpreted in \(\mathbb{Q}\).
\leancodefile[firstline=12,lastline=20,firstnumber=12]{../../ClassificationOfIntegersOfQuadraticNumberFields/ClassificationToZsqrtd.lean}

\subsection{Theorem 6.2 (injectivity)}
\textbf{Lean name:} \texttt{zsqrtdToQsqrtd\_injective}.\newline
\[
\iota_d(z_1)=\iota_d(z_2)\implies z_1=z_2.
\]
\leancodefile[firstline=23,lastline=27,firstnumber=23]{../../ClassificationOfIntegersOfQuadraticNumberFields/ClassificationToZsqrtd.lean}

\subsection{Definition 6.3 (equivalence with image)}
\textbf{Lean name:} \texttt{zsqrtdEquivRangeQsqrtd}.\newline
There is a ring isomorphism
\[
\mathbb{Z}[\sqrt d]\cong \operatorname{im}(\iota_d).
\]
\leancodefile[firstline=30,lastline=38,firstnumber=30]{../../ClassificationOfIntegersOfQuadraticNumberFields/ClassificationToZsqrtd.lean}

\subsection{Theorem 6.4 (half-integral image criterion)}
\textbf{Lean name:} \texttt{halfInt\_mem\_range\_zsqrtdToQsqrtd\_iff\_even\_even}.\newline
For integers \(a',b'\),
\[
\frac{a'+b'\sqrt d}{2}\in \operatorname{im}(\iota_d)
\iff
2\mid a'\ \text{and}\ 2\mid b'.
\]
\leancodefile[firstline=41,lastline=62,firstnumber=41]{../../ClassificationOfIntegersOfQuadraticNumberFields/ClassificationToZsqrtd.lean}

\subsection{Theorem 6.5 (classification in \(d\not\equiv 1\mod 4\) branch)}
\textbf{Lean name:} \texttt{dvd\_four\_sub\_sq\_iff\_exists\_zsqrtd\_image\_of\_ne\_one\_mod\_four}.\newline
If \(d\) is squarefree and \(d\not\equiv 1\pmod 4\), then
\[
4\mid(a'^2-d b'^2)
\iff
\exists z\in\mathbb{Z}[\sqrt d],\;\iota_d(z)=\frac{a'+b'\sqrt d}{2}.
\]
This gives the completed branch of the quadratic-integer classification proof.
\leancodefile[firstline=41,lastline=70,firstnumber=41]{../../ClassificationOfIntegersOfQuadraticNumberFields/ClassificationToZsqrtd.lean}

\section{Non-isomorphism of distinct quadratic fields (NonIso.lean)}

\subsection{Lemma 7.1}
\textbf{Lean name:} \texttt{not\_isSquare\_neg\_one\_rat}.\newline
\[
-1\ \text{is not a square in }\mathbb{Q}.
\]
\leancodefile[firstline=8,lastline=11,firstnumber=8]{../../ClassificationOfIntegersOfQuadraticNumberFields/NonIso.lean}

\subsection{Lemma 7.2}
\textbf{Lean name:} \texttt{nat\_eq\_one\_of\_squarefree\_intcast\_of\_isSquare}.\newline
If \(m\in\mathbb{N}\), \((m:\mathbb{Z})\) is squarefree, and \((m:\mathbb{Z})\) is a square,
then
\[
m=1.
\]
\leancodefile[firstline=14,lastline=28,firstnumber=14]{../../ClassificationOfIntegersOfQuadraticNumberFields/NonIso.lean}

\subsection{Lemma 7.3}
\textbf{Lean name:} \texttt{int\_dvd\_of\_ratio\_square}.\newline
Let \(d_2\neq 0\), with \(d_2\) squarefree. If
\[
\frac{d_1}{d_2}\in\mathbb{Q}
\]
is a square in \(\mathbb{Q}\), then
\[
d_2\mid d_1.
\]
\leancodefile[firstline=31,lastline=44,firstnumber=31]{../../ClassificationOfIntegersOfQuadraticNumberFields/NonIso.lean}

\subsection{Theorem 7.4 (distinct parameters give non-isomorphic fields)}
\textbf{Lean name:} \texttt{quadratic\_fields\_not\_iso}.\newline
Assume \(d_1,d_2\) satisfy \texttt{IsQuadraticParam} and \(d_1\neq d_2\). Then
\[
\mathbb{Q}(\sqrt{d_1})\not\cong_{\mathbb{Q}}\mathbb{Q}(\sqrt{d_2}).
\]
The proof follows a standard reduction: an assumed isomorphism forces square-ratio
conditions implying divisibility both ways; associatedness yields either equality or sign
flip; the sign-flip branch reduces to \(-1\) being a rational square, contradiction.
\leancodefile[firstline=46,lastline=117,firstnumber=46]{../../ClassificationOfIntegersOfQuadraticNumberFields/NonIso.lean}

\section{Progress and remaining branch}
The formalization now fully covers the parity/divisibility mechanism and the
\(d\not\equiv 1\pmod 4\) classification branch. The explicit open item in source is the
\(d\equiv 1\pmod 4\) structural branch, expected to use
\(\mathbb{Z}[\frac{1+\sqrt d}{2}]\) as the integral model.

\subsection*{Planned refactor direction}
The long-term plan is to reduce reliance on the coarse ambient path and move to a
cleaner integral-first architecture:
\begin{enumerate}
  \item Refactor the bridge around \texttt{Zsqrtd}-centric algebraic interfaces where possible.
  \item Add or formalize the missing
  \(\mathbb{Z}\!\left[\frac{1+\sqrt{1+4k}}{2}\right]\)-style construction needed for
  the \(1\pmod 4\) branch.
  \item Unify both congruence branches into a final ring-of-integers classification theorem
  with a single API-level statement.
\end{enumerate}
So the present report should be read as a robust intermediate stage: the core arithmetic
lemmas are already in place, and the next phase is structural cleanup plus completion of
the missing integral model.

\section{Reproducibility}
\begin{verbatim}
lake exe cache get
lake build
cd tex/report
latexmk -xelatex -shell-escape -interaction=nonstopmode -halt-on-error -output-directory=out report.tex
\end{verbatim}

\end{document}
