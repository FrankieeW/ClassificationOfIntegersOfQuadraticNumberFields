% Copyright (c) 2026 Frankie Feng-Cheng WANG. All rights reserved.
% Repository: https://github.com/FrankieeW/ClassificationOfIntegersOfQuadraticNumberFields

\documentclass[
  12pt,
]{assignment}

\setcounter{tocdepth}{2}
\setcounter{secnumdepth}{2}

\usepackage{amsthm}
\setmonofont{FreeMono}
\usepackage{cleveref}
\newcommand{\leancodefile}[2][]{%
  \inputminted[
    fontsize=\footnotesize,
    breaklines=true,
    numbers=left,
    numbersep=8pt,
    tabsize=4,
    #1
  ]{lean4}{#2}%
}

\setlength{\headheight}{30.56006pt}
\addtolength{\topmargin}{-18.56006pt}

\title{Quadratic Integer Rings in Lean 4\\\large Formalization Report}
\author{Frankie Feng-Cheng WANG}
\email{maths@frankie.wang}
\github{https://github.com/FrankieeW/ClassificationOfIntegersOfQuadraticNumberFields}
\date{\today}
\institute{Department of Mathematics\\Imperial College London}
\course{MATH70040-Formalising Mathematics}
\lecturer{Dr Bhavik Mehta}

\begin{document}
\maketitle
\tableofcontents

\section{Introduction}
This report documents a Lean~4 formalization of results
from the algebraic-integers chapter (Lecture~2) of Boxer's
notes~\cite{boxer2024algebraicnt}, building on Mathlib~\cite{mathlib4}.
Every definition, lemma, and theorem listed below has been formally verified;
the corresponding Lean source is included inline for reference.

The mathematical scope is the quadratic field
\[
\mathbb{Q}(\sqrt d)=\{a+b\sqrt d\mid a,b\in\mathbb{Q}\},
\]
and the ring-of-integers prerequisites needed for the classification split by
\(d\bmod 4\).
The central goal is to formalize the standard result
(cf.~\cite[Lecture~2]{boxer2024algebraicnt}):
\[
\mathcal{O}_{\mathbb{Q}(\sqrt d)}=
\begin{cases}
\mathbb{Z}[\sqrt d] & \text{if } d\not\equiv 1\pmod 4,\\[4pt]
\mathbb{Z}\!\left[\dfrac{1+\sqrt d}{2}\right] & \text{if } d\equiv 1\pmod 4.
\end{cases}
\]

\noindent\textbf{Structure.}
\cref{sec:setup} sets up the ambient quadratic field.
\cref{sec:tracenorm} records the trace, norm, and minimal-polynomial identity.
\cref{sec:noniso} proves non-isomorphism of distinct quadratic fields.
\cref{sec:halfint,sec:modfour} develop the half-integral normal form and the mod-4
parity criterion.
\cref{sec:classification} combines these into the element-level classification for
the \(d\not\equiv 1\pmod 4\) branch.

\noindent\textbf{Build status (\today):}
\texttt{lake build} succeeds.
Two \texttt{sorry} markers remain in \texttt{ClassificationToZsqrtd.lean}\footnote{It is not in submission ZIP}
(the forward and reverse directions linking the ring-of-integers isomorphism
to the mod-4 condition); all other files are sorry-free.

\subsection*{Development note (motivation and current strategy)}
The current formalization strategy is intentionally pragmatic.
Historically, while preparing and submitting PR work around \texttt{Zsqrtd},
I noticed that \texttt{QuadraticAlgebra} provides a workable ambient path for this
project stage.
So for the half-integral classification workflow I currently use
\texttt{QuadraticAlgebra} and define
\[
\texttt{Qsqrtd}(d):=\texttt{QuadraticAlgebra}\;\mathbb{Q}\;(d:\mathbb{Q})\;0.
\]
This is a coarse-grained but effective bridge for now.
The reason is that the integral model needed for the
\(d\equiv 1\pmod 4\) branch, especially the
\(\mathbb{Z}[\frac{1+\sqrt{1+4k}}{2}]\)-style object, is not yet packaged as a ready
drop-in component in the way this project needs; see the related discussion:
\href{https://leanprover.zulipchat.com/\#narrow/channel/217875-Is-there-code-for-X.3F/topic/Z.5B.281.2Bsqrt.281.2B4k.29.29.2F2.5D/near/520523635}{Zulip discussion}.

%% ─── Section 2 ─────────────────────────────────────────────────────────────────
\section{Quadratic setup and basic structures (Base.lean)}\label{sec:setup}
This section sets up the type-level infrastructure for working with
\(\mathbb{Q}(\sqrt d)\).
Following~\cite[Lecture~2, \S1]{boxer2024algebraicnt}, we first pin down the
admissible parameters \(d\), then construct the ambient field together with its
trace, norm, and embedding.

\subsection{Definition 2.1 (quadratic parameter package)}
\textbf{Lean name:} \texttt{IsQuadraticParam}.\newline
For \(d\in\mathbb{Z}\), we define a proposition requiring
\[
d\neq 0,\qquad d\neq 1,\qquad \text{$d$ squarefree}.
\]
These are the standard hypotheses ensuring that \(\mathbb{Q}(\sqrt d)\) is a genuine
quadratic extension of~\(\mathbb{Q}\)
(cf.~\cite[Lecture~2, Definition~2.1]{boxer2024algebraicnt}).
\leancodefile[firstline=21,lastline=26,firstnumber=21]{../../ClassificationOfIntegersOfQuadraticNumberFields/Base.lean}

\subsection{Definition 2.2 (ambient type)}
\textbf{Lean name:} \texttt{Qsqrtd}.\newline
The type
\[
\texttt{Qsqrtd}(d):=\texttt{QuadraticAlgebra}\;\mathbb{Q}\;(d: \mathbb{Q})\;0
\]
serves as the formal model of \(\mathbb{Q}(\sqrt d)\).
\leancodefile[firstline=29,lastline=29,firstnumber=29]{../../ClassificationOfIntegersOfQuadraticNumberFields/Base.lean}

\subsection{Definition 2.3 (rescaling equivalence)}
\textbf{Lean name:} \texttt{rescale}.\newline
Given \(a\in\mathbb{Q}^{\times}\), there is an algebra isomorphism
\[
\mathbb{Q}(\sqrt d)\cong\mathbb{Q}(\sqrt{a^2d}).
\]
In coordinates this is
\[
(r,s)\longmapsto (r,s a^{-1}),\qquad (r,t)\longmapsto (r,ta).
\]
This captures the classical fact that replacing \(d\) by \(a^2 d\) does not change
the underlying quadratic field.
\leancodefile[firstline=35,lastline=50,firstnumber=35]{../../ClassificationOfIntegersOfQuadraticNumberFields/Base.lean}

\subsection{Definition 2.4 (trace and norm abbreviations)}
\textbf{Lean names:} \texttt{trace}, \texttt{norm'}.\newline
For \(x\in \texttt{Qsqrtd}(d)\), define
\[
\operatorname{tr}(x):=x+\bar x\in\mathbb{Q},\qquad N(x):=x\bar x\in\mathbb{Q}.
\]
These correspond to the trace and norm of the quadratic extension
\(\mathbb{Q}(\sqrt d)/\mathbb{Q}\), computed via the Galois conjugation
\(\sqrt d\mapsto -\sqrt d\).
\leancodefile[firstline=53,lastline=56,firstnumber=53]{../../ClassificationOfIntegersOfQuadraticNumberFields/Base.lean}

\subsection{Definition 2.5 (rational embedding)}
\textbf{Lean name:} \texttt{embed}.\newline
The canonical inclusion
\[
\mathbb{Q}\hookrightarrow \mathbb{Q}(\sqrt d)
\]
is implemented by the algebra map.
\leancodefile[firstline=59,lastline=59,firstnumber=59]{../../ClassificationOfIntegersOfQuadraticNumberFields/Base.lean}

\subsection{Definition 2.6 (nonsquare rational condition)}
\textbf{Lean name:} \texttt{IsNonsquareRat}.\newline
For integer \(d\), define
\[
\forall r\in\mathbb{Q},\quad r^2\neq d.
\]
\leancodefile[firstline=62,lastline=63,firstnumber=62]{../../ClassificationOfIntegersOfQuadraticNumberFields/Base.lean}

\subsection{Proposition 2.7 (squarefree nontrivial implies nonsquare in \texorpdfstring{$\mathbb{Z}$}{Z})}
\textbf{Lean name:} \texttt{not\_isSquare\_int}.\newline
Under \texttt{IsQuadraticParam} hypotheses,
\[
\neg\,\mathrm{IsSquare}(d)\quad\text{in }\mathbb{Z}.
\]
This excludes degeneration of the quadratic extension.
\leancodefile[firstline=66,lastline=79,firstnumber=66]{../../ClassificationOfIntegersOfQuadraticNumberFields/Base.lean}

\subsection{Proposition 2.8 (parameter hypothesis gives rational nonsquare)}
\textbf{Lean instance:} \texttt{instance (d) [IsQuadraticParam d] : IsNonsquareRat d}.\newline
From the integer nonsquare result and transfer lemmas between \(\mathbb{Z}\) and
\(\mathbb{Q}\), one obtains
\[
\forall r\in\mathbb{Q},\;r^2\neq d.
\]
\leancodefile[firstline=81,lastline=86,firstnumber=81]{../../ClassificationOfIntegersOfQuadraticNumberFields/Base.lean}

\subsection{Proposition 2.9 (field structure)}
\textbf{Lean instance:} \texttt{Field (Qsqrtd d)} under \texttt{IsNonsquareRat d}.\newline
If \(d\) is rationally nonsquare, then \(\mathbb{Q}(\sqrt d)\) is a field.
This is the key structural fact that upgrades the ring \(\texttt{Qsqrtd}(d)\) to a
number field.
\leancodefile[firstline=88,lastline=93,firstnumber=88]{../../ClassificationOfIntegersOfQuadraticNumberFields/Base.lean}

%% ─── Section 3 ─────────────────────────────────────────────────────────────────
\section{Trace, norm, and quadratic identity (MinimalPolynomial.lean)}\label{sec:tracenorm}
With the ambient field in place, we now establish the explicit trace and norm
formulas and verify the characteristic polynomial identity.
These are the algebraic prerequisites for the integrality criterion used in
\cref{sec:modfour}.

\subsection{Theorem 3.1 (trace formula)}
\textbf{Lean name:} \texttt{trace\_eq\_two\_re}.\newline
For \(x=a+b\sqrt d\in\mathbb{Q}(\sqrt d)\),
\[
\operatorname{tr}(x)=2a.
\]
\leancodefile[firstline=8,lastline=12,firstnumber=8]{../../ClassificationOfIntegersOfQuadraticNumberFields/MinimalPolynomial.lean}

\subsection{Theorem 3.2 (norm formula)}
\textbf{Lean name:} \texttt{norm\_eq\_sqr\_minus\_d\_sqr}.\newline
Writing \(x=a+b\sqrt d\), one has
\[
N(x)=a^2-d b^2.
\]
\leancodefile[firstline=15,lastline=19,firstnumber=15]{../../ClassificationOfIntegersOfQuadraticNumberFields/MinimalPolynomial.lean}

\subsection{Theorem 3.3 (quadratic polynomial annihilation)}
\textbf{Lean name:} \texttt{aeval\_eq\_zero\_of\_quadratic}.\newline
Each \(x\in\mathbb{Q}(\sqrt d)\) satisfies
\[
x^2-\operatorname{tr}(x)\,x+N(x)=0.
\]
In other words, every element of the quadratic extension is a root of the polynomial
\(T^2 - \operatorname{tr}(x)\,T + N(x)\), which is its minimal polynomial over
\(\mathbb{Q}\) when \(x\notin\mathbb{Q}\).
\leancodefile[firstline=21,lastline=24,firstnumber=21]{../../ClassificationOfIntegersOfQuadraticNumberFields/MinimalPolynomial.lean}

%% ─── Section 4 ─────────────────────────────────────────────────────────────────
\section{Non-isomorphism of distinct quadratic fields (NonIso.lean)}\label{sec:noniso}
Before turning to the ring-of-integers classification, we address a natural
prerequisite: distinct squarefree parameters really do give distinct fields.
This section formalizes the classical result that if \(d_1\neq d_2\) are
both valid quadratic parameters, then \(\mathbb{Q}(\sqrt{d_1})\) and
\(\mathbb{Q}(\sqrt{d_2})\) are not isomorphic as \(\mathbb{Q}\)-algebras.

\subsection{Lemma 4.1}
\textbf{Lean name:} \texttt{not\_isSquare\_neg\_one\_rat}.\newline
\[
-1\ \text{is not a square in }\mathbb{Q}.
\]
\leancodefile[firstline=8,lastline=11,firstnumber=8]{../../ClassificationOfIntegersOfQuadraticNumberFields/NonIso.lean}

\subsection{Lemma 4.2}
\textbf{Lean name:} \texttt{nat\_eq\_one\_of\_squarefree\_intcast\_of\_isSquare}.\newline
If \(m\in\mathbb{N}\), \((m:\mathbb{Z})\) is squarefree, and \((m:\mathbb{Z})\) is a square,
then
\[
m=1.
\]
\leancodefile[firstline=14,lastline=28,firstnumber=14]{../../ClassificationOfIntegersOfQuadraticNumberFields/NonIso.lean}

\subsection{Lemma 4.3}
\textbf{Lean name:} \texttt{int\_dvd\_of\_ratio\_square}.\newline
Let \(d_2\neq 0\), with \(d_2\) squarefree. If
\[
\frac{d_1}{d_2}\in\mathbb{Q}
\]
is a square in \(\mathbb{Q}\), then
\[
d_2\mid d_1.
\]
This is the key divisibility extraction used in the non-isomorphism argument.
\leancodefile[firstline=31,lastline=44,firstnumber=31]{../../ClassificationOfIntegersOfQuadraticNumberFields/NonIso.lean}

\subsection{Theorem 4.4 (distinct parameters give non-isomorphic fields)}
\textbf{Lean name:} \texttt{quadratic\_fields\_not\_iso}.\newline
Assume \(d_1,d_2\) satisfy \texttt{IsQuadraticParam} and \(d_1\neq d_2\). Then
\[
\mathbb{Q}(\sqrt{d_1})\not\cong_{\mathbb{Q}}\mathbb{Q}(\sqrt{d_2}).
\]
The proof follows a standard reduction: an assumed isomorphism forces square-ratio
conditions implying divisibility both ways (via Lemma~4.3); associatedness yields
either equality or sign flip; the sign-flip branch reduces to \(-1\) being a rational
square (Lemma~4.1), contradiction.
\leancodefile[firstline=46,lastline=117,firstnumber=46]{../../ClassificationOfIntegersOfQuadraticNumberFields/NonIso.lean}

%% ─── Section 5 ─────────────────────────────────────────────────────────────────
\section{Half-integral normal form (HalfInt.lean)}\label{sec:halfint}
Every element of \(\mathbb{Q}(\sqrt d)\) that is integral over \(\mathbb{Z}\)
can be written in the form \(\frac{a'+b'\sqrt d}{2}\) with \(a',b'\in\mathbb{Z}\)
(cf.~\cite[Lecture~2, proof of Theorem~2.5]{boxer2024algebraicnt}).
This section sets up the half-integral representation and computes its trace and norm
explicitly.

\subsection{Definition 5.1}
\textbf{Lean name:} \texttt{halfInt}.\newline
For integers \(a',b',d\), define
\[
\operatorname{halfInt}(d,a',b'):=\frac{a'+b'\sqrt d}{2}\in\mathbb{Q}(\sqrt d).
\]
\leancodefile[firstline=8,lastline=9,firstnumber=8]{../../ClassificationOfIntegersOfQuadraticNumberFields/HalfInt.lean}

\subsection{Theorem 5.2 (trace of half-integral element)}
\textbf{Lean name:} \texttt{trace\_halfInt}.\newline
\[
\operatorname{tr}\!\left(\frac{a'+b'\sqrt d}{2}\right)=a'.
\]
\leancodefile[firstline=12,lastline=16,firstnumber=12]{../../ClassificationOfIntegersOfQuadraticNumberFields/HalfInt.lean}

\subsection{Theorem 5.3 (norm of half-integral element)}
\textbf{Lean name:} \texttt{norm\_halfInt}.\newline
\[
N\!\left(\frac{a'+b'\sqrt d}{2}\right)=\frac{a'^2-d b'^2}{4}.
\]
An element \(\frac{a'+b'\sqrt d}{2}\) is integral over \(\mathbb{Z}\) if and only if
both \(\operatorname{tr}\) and \(N\) are integers, which by the formulas above
reduces to \(a'\in\mathbb{Z}\) (automatic) and \(4\mid(a'^2-db'^2)\).
This divisibility condition is the starting point for the mod-4 analysis in
\cref{sec:modfour}.
\leancodefile[firstline=18,lastline=24,firstnumber=18]{../../ClassificationOfIntegersOfQuadraticNumberFields/HalfInt.lean}

%% ─── Section 6 ─────────────────────────────────────────────────────────────────
\section{Mod-4 analysis and parity classification (ModFourCriteria.lean)}\label{sec:modfour}
This section reduces the integrality condition \(4\mid(a'^2-db'^2)\) to explicit
congruence constraints on \(a'\), \(b'\), and \(d\).
The analysis mirrors the key exercise pattern from
\cite[Lecture~2]{boxer2024algebraicnt}: a case split on the parities of \(a'\) and
\(b'\), combined with the mod-4 residue of~\(d\).

\subsection{Lemma 6.1}
\textbf{Lean name:} \texttt{squarefree\_int\_not\_dvd\_four}.\newline
If \(d\in\mathbb{Z}\) is squarefree, then
\[
4\nmid d.
\]
Indeed, \(4 = 2^2\) dividing~\(d\) would contradict squarefreeness.
\leancodefile[firstline=9,lastline=15,firstnumber=9]{../../ClassificationOfIntegersOfQuadraticNumberFields/ModFourCriteria.lean}

\subsection{Lemma 6.2}
\textbf{Lean name:} \texttt{squarefree\_int\_emod\_four}.\newline
If \(d\) is squarefree, then
\[
d\bmod 4\in\{1,2,3\}.
\]
This is an immediate corollary of Lemma~6.1.
\leancodefile[firstline=18,lastline=21,firstnumber=18]{../../ClassificationOfIntegersOfQuadraticNumberFields/ModFourCriteria.lean}

\subsection{Lemma 6.3}
\textbf{Lean name:} \texttt{Int.sq\_emod\_four\_of\_even}.\newline
If \(2\mid n\), then
\[
n^2\equiv 0\pmod 4.
\]
\leancodefile[firstline=24,lastline=27,firstnumber=24]{../../ClassificationOfIntegersOfQuadraticNumberFields/ModFourCriteria.lean}

\subsection{Lemma 6.4}
\textbf{Lean name:} \texttt{Int.sq\_emod\_four\_of\_odd}.\newline
If \(2\nmid n\), then
\[
n^2\equiv 1\pmod 4.
\]
Lemmas~6.3 and~6.4 together show that the mod-4 residue of a square is determined
entirely by its parity, which is the key arithmetic input for the main criterion.
\leancodefile[firstline=30,lastline=35,firstnumber=30]{../../ClassificationOfIntegersOfQuadraticNumberFields/ModFourCriteria.lean}

\subsection{Lemma 6.5 (internal equivalence)}
\textbf{Lean name:} \texttt{div4\_iff\_mod} (private).\newline
For integers \(a',b',d\),
\[
4\mid(a'^2-d b'^2)\iff (a'^2-d b'^2)\bmod 4=0.
\]
\leancodefile[firstline=37,lastline=39,firstnumber=37]{../../ClassificationOfIntegersOfQuadraticNumberFields/ModFourCriteria.lean}

\subsection{Theorem 6.6 (main mod-4 criterion)}\label{thm:main-mod4}
\textbf{Lean name:} \texttt{dvd\_four\_sub\_sq\_iff\_even\_even\_or\_odd\_odd\_mod\_four\_one}.\newline
Assume \(d\) squarefree. Then
\[
4\mid(a'^2-d b'^2)
\iff
\bigl(2\mid a'\ \&\ 2\mid b'\bigr)
\;\vee\;
\bigl(2\nmid a'\ \&\ 2\nmid b'\ \&\ d\equiv 1\pmod 4\bigr).
\]
The proof proceeds by exhaustive case analysis on the parities of \(a'\) and \(b'\):
the mixed-parity cases are ruled out by Lemmas~6.3--6.4 and the constraint
\(d\bmod 4\in\{1,2,3\}\); the odd--odd case forces \(d\equiv 1\pmod 4\).
\leancodefile[firstline=42,lastline=101,firstnumber=42]{../../ClassificationOfIntegersOfQuadraticNumberFields/ModFourCriteria.lean}

\subsection{Theorem 6.7 (forcing even--even when \texorpdfstring{$d\not\equiv 1\bmod 4$}{d is not 1 mod 4})}
\textbf{Lean name:} \texttt{even\_even\_of\_dvd\_four\_sub\_sq\_of\_ne\_one\_mod\_four}.\newline
If \(d\) is squarefree and \(d\not\equiv 1\pmod 4\), then
\[
4\mid(a'^2-d b'^2)\implies 2\mid a'\ \text{and}\ 2\mid b'.
\]
\leancodefile[firstline=104,lastline=110,firstnumber=104]{../../ClassificationOfIntegersOfQuadraticNumberFields/ModFourCriteria.lean}

\subsection{Theorem 6.8 (equivalence in non-\texorpdfstring{$1\bmod 4$}{1 mod 4} branch)}
\textbf{Lean name:} \texttt{dvd\_four\_sub\_sq\_iff\_even\_even\_of\_ne\_one\_mod\_four}.\newline
If \(d\) is squarefree and \(d\not\equiv 1\pmod 4\), then
\[
4\mid(a'^2-d b'^2)\iff \bigl(2\mid a'\ \&\ 2\mid b'\bigr).
\]
\leancodefile[firstline=113,lastline=120,firstnumber=113]{../../ClassificationOfIntegersOfQuadraticNumberFields/ModFourCriteria.lean}

\subsection{Theorem 6.9 (equivalence in \texorpdfstring{$1\bmod 4$}{1 mod 4} branch)}
\textbf{Lean name:} \texttt{dvd\_four\_sub\_sq\_iff\_same\_parity\_of\_one\_mod\_four}.\newline
If \(d\) is squarefree and \(d\equiv 1\pmod 4\), then
\[
4\mid(a'^2-d b'^2)\iff a'\equiv b'\pmod 2.
\]
Together with \cref{thm:main-mod4}, Theorems~6.8 and~6.9 give the complete
parity characterization: in the \(d\not\equiv 1\) branch only the
\enquote{both even} case survives, while in the \(d\equiv 1\) branch the
\enquote{same parity} condition captures both the even--even and odd--odd cases.
\leancodefile[firstline=122,lastline=135,firstnumber=122]{../../ClassificationOfIntegersOfQuadraticNumberFields/ModFourCriteria.lean}

%% ─── Section 7 ─────────────────────────────────────────────────────────────────
\section{Embedding into \texorpdfstring{$\mathbb{Q}(\sqrt d)$}{Q(sqrt d)} and image characterization (ClassificationToZsqrtd.lean)}\label{sec:classification}
With the mod-4 arithmetic in hand, we now connect it to the algebraic structure.
The strategy is to embed Mathlib's \(\mathbb{Z}[\sqrt d]\) (the type
\texttt{Zsqrtd}) into our ambient field model and characterize its image in terms
of the half-integral representation from \cref{sec:halfint}.

\subsection{Definition 7.1 (canonical embedding)}
\textbf{Lean name:} \texttt{zsqrtdToQsqrtd}.\newline
Define the ring map
\[
\iota_d:\mathbb{Z}[\sqrt d]\longrightarrow \mathbb{Q}(\sqrt d),
\qquad m+n\sqrt d\longmapsto m+n\sqrt d,
\]
with coefficients interpreted in \(\mathbb{Q}\).
\leancodefile[firstline=12,lastline=20,firstnumber=12]{../../ClassificationOfIntegersOfQuadraticNumberFields/ClassificationToZsqrtd.lean}

\subsection{Theorem 7.2 (injectivity)}
\textbf{Lean name:} \texttt{zsqrtdToQsqrtd\_injective}.\newline
\[
\iota_d(z_1)=\iota_d(z_2)\implies z_1=z_2.
\]
\leancodefile[firstline=23,lastline=27,firstnumber=23]{../../ClassificationOfIntegersOfQuadraticNumberFields/ClassificationToZsqrtd.lean}

\subsection{Definition 7.3 (equivalence with image)}
\textbf{Lean name:} \texttt{zsqrtdEquivRangeQsqrtd}.\newline
There is a ring isomorphism
\[
\mathbb{Z}[\sqrt d]\cong \operatorname{im}(\iota_d).
\]
\leancodefile[firstline=30,lastline=38,firstnumber=30]{../../ClassificationOfIntegersOfQuadraticNumberFields/ClassificationToZsqrtd.lean}

\subsection{Theorem 7.4 (half-integral image criterion)}
\textbf{Lean name:} \texttt{halfInt\_mem\_range\_zsqrtdToQsqrtd\_iff\_even\_even}.\newline
For integers \(a',b'\),
\[
\frac{a'+b'\sqrt d}{2}\in \operatorname{im}(\iota_d)
\iff
2\mid a'\ \text{and}\ 2\mid b'.
\]
This is the bridge between the half-integral normal form and the
\(\mathbb{Z}[\sqrt d]\)-representability question.
\leancodefile[firstline=41,lastline=62,firstnumber=41]{../../ClassificationOfIntegersOfQuadraticNumberFields/ClassificationToZsqrtd.lean}

\subsection{Theorem 7.5 (classification in \texorpdfstring{$d\not\equiv 1\bmod 4$}{d not equiv 1 mod 4} branch)}
\textbf{Lean name:} \texttt{dvd\_four\_sub\_sq\_iff\_exists\_zsqrtd\_image\_of\_ne\_one\_mod\_four}.\newline
If \(d\) is squarefree and \(d\not\equiv 1\pmod 4\), then
\[
4\mid(a'^2-d b'^2)
\iff
\exists z\in\mathbb{Z}[\sqrt d],\;\iota_d(z)=\frac{a'+b'\sqrt d}{2}.
\]
Combining the mod-4 criterion (Theorem~6.8) with the image criterion
(Theorem~7.4), this establishes that an integral element in half-integral form
lies in \(\mathbb{Z}[\sqrt d]\) precisely when \(d\not\equiv 1\pmod 4\).
This completes the \(d\not\equiv 1\) branch of the classification at the
element-level.
\leancodefile[firstline=68,lastline=73,firstnumber=68]{../../ClassificationOfIntegersOfQuadraticNumberFields/ClassificationToZsqrtd.lean}


%% ─── Section 8 ─────────────────────────────────────────────────────────────────
\section{Progress and remaining work}
\subsection*{Current progress}
The formalization now fully covers:
\begin{itemize}
  \item The quadratic field setup and basic algebraic infrastructure
    (\cref{sec:halfint} and preceding sections).
  \item The complete parity/divisibility mechanism reducing integrality to
    congruence conditions (\cref{sec:modfour}).
  \item The embedding and image characterization of \(\mathbb{Z}[\sqrt d]\)
    inside \(\mathbb{Q}(\sqrt d)\), completing the element-level classification
    for the \(d\not\equiv 1\pmod 4\) branch.
  \item The non-isomorphism theorem for distinct quadratic fields.
\end{itemize}

\subsection*{Remaining (not yet formalized)}
The explicit open item is the \(d\equiv 1\pmod 4\) structural branch,
expected to use \(\mathbb{Z}[\frac{1+\sqrt d}{2}]\) as the integral model.
Concretely, the next milestones are:
\begin{enumerate}
  \item Formalize the missing
  \(\mathbb{Z}\!\left[\frac{1+\sqrt{1+4k}}{2}\right]\)-style construction needed for
  the \(1\pmod 4\) branch.
  \item Complete the two remaining \texttt{sorry} markers in
  \texttt{ClassificationToZsqrtd.lean}, linking the element-level classification
  to the ring-of-integers isomorphism.
  \item Unify both congruence branches into a final ring-of-integers classification
  theorem with a single API-level statement.
\end{enumerate}

\printbibliography

\end{document}
